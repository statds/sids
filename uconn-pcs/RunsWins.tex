\documentclass[12pt]{article}

\usepackage{amssymb, amsfonts, amsmath, amsthm, enumitem}
\usepackage{hyperref}

\usepackage{fullpage}

\def\dd{\mathrm{d}}
\def\E{\mathrm{E}}


\begin{document}
\begin{center}
  \bf Possible Project: Runs and Wins of Baseball\\
  UConn--PCS: Data Science, 2021\\
  Due Date: 11 am, July 1, 2021
\end{center}

The percentage of wins obtained by a team over the course of a season
is strongly related with the number of runs it scores and allows. The
relationship between runs and wins is explored in Chapter~4 of the
book by
\href{https://www.amazon.com/gp/product/B07KRNP2BB/ref=dbs_a_def_rwt_bibl_vppi_i0}{Marchi,
  Albert, and Baumer (2019)}.
The source code of the book is at
\url{https://github.com/bayesball/bayesball.github.io}.
A workshop on baseball analytics at the last two UConn Sports
Analytics Symposiums led by Dr. Zhe Wang included this as an
example. The training materials of this workshop can be accessed at
\url{https://statds.org/events/ucsas2020/workshops.html#baseball}.
In particular, after downloading it, the code for this example is
\texttt{Runs\_and\_Wins.R} under the \texttt{Code} folder.

\begin{enumerate}
\item
Run the example code on your own computer line by line to reproduce
the analysis.

\item
Explain what Pythagorean formula is and how well it fits baseball.

\item
Follow your interests to further explore the data and report your
findings. Examples topics are: relationship between win percentage and
run differential across decades; Pythagorean residuals for stronger
and weaker teams. Be creative and don't be limited by the
examples.

\end{enumerate}

\end{document}
