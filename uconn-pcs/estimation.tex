\documentclass[leqno]{beamer}

\usetheme{Storrs}
\usecolortheme{lily}
\useinnertheme{rounded}

\usepackage{setspace, graphicx,hyperref}
\usepackage{amsfonts,amsthm,amsmath}
\usepackage{caption}
\usepackage{color}
\usepackage{multicol}
\usepackage{textpos}
\usepackage{tikz}
\usepackage[authoryear]{natbib}
\usepackage[linesnumbered,algoruled,boxed,lined]{algorithm2e}
\hypersetup{
	colorlinks,
	citecolor=blue}
\usepackage{multirow}
\usepackage{booktabs}
\usepackage{multicol}
\usepackage{soul} %% striketrhough texts
\usepackage{bm}

\newtheorem{proposition}{Proposition}

% \usetheme{Madrid}
% \usecolortheme{seahorse}

%% notations
\newcommand{\E}{\mathbb{E}\,}
\newcommand{\V}{\mathbb{V}}

\newcommand{\blue}[1]{\textcolor{blue}{#1}}
\newcommand{\red}[1]{\textcolor{red}{#1}}

\usepackage{graphicx}
% graphic path
\graphicspath{{./}{../manuscript/image/}}

\setbeamercovered{transparent}
\setbeamertemplate{itemize item}[circle]
\setbeamertemplate{enumerate item}[default]
\setbeamertemplate{section in toc}[default]
\setbeamertemplate{subsection in toc}[default]
\setbeamertemplate{caption}[numbered]

% \AtBeginSection[]{ 
% 	\begin{frame}<beamer>{Overview} 
% 	\tableofcontents[currentsection] 
% \end{frame}} 

\title[Estimation]{
  Estimation: A hide-and-seek game with the nature}
\subtitle[]{UConn-PCS 2021, Data Science}
\author[Jun Yan]{Jun Yan$^{1,2,3}$}
\institute[UConn]{
$^{1}$Department of Statistics, University of Connecticut\\
$^{2}$Center for Population Health, UConn Health\\
$^{3}$Center for Environmental Sciences and Engineering, University of Connecticut
}

\date{June 23, 2021 @ UConn}

\begin{document}

\begin{frame}[plain]
\titlepage
\end{frame}


\begin{frame}
  \frametitle{Hide-and-Seek}
  \begin{itemize}
  \item
    Examples (from my own research)
    \begin{itemize}
    \item Survival time of lymphoma cancer patients
    \item Climate change attributable to human activities
    \item Time a mountain lion spend in moving/resting
    \item Which NCAA team should rank the first
    \end{itemize}
  \item
    This is a game: the nature hides something that we are interested
    in; we want to find it with some clues (inference!)
  \item
    Statisticians/data scientists are detectives: reveal the unknown
    truth from observed data.
  \end{itemize}
\end{frame}


\begin{frame}
  \frametitle{A Simple Setup}
  \begin{itemize}
  \item We observe a random sample $X_1, \ldots, X_n$ from a
    population, which is characterized by an unknown location
    parameter $\theta$.
    \begin{itemize}
    \item A location parameter shifts the location of a distribution.
    \end{itemize}

  \item Population examples:
    Normal with mean/location $\theta$ and variance~1;    
    Cauchy with location $\theta$ and scale~1.

  \item Statistics: summaries from the observed sample.
    \begin{itemize}
    \item A statistics is computable from a sample (which contains no
      unknown information).
    \item Sample moments are statistics.
    \item An estimator is a statistic.
    \end{itemize}    
  \end{itemize}
\end{frame}


\begin{frame}
  \frametitle{Estimating the Location of a Symmetric Distribution}
  \begin{itemize}
  \item
    Candidate estimators $\hat\theta$
    \begin{itemize}
    \item Sample mean
    \item Sample median
    \item Mid sample range
    % \item Trimmed mean
    % \item Winsorized mean
    \end{itemize}
  \item
    A realization of an estimator is called an estimate.
  \item
    Performance in one specific sample does not mean much.
  \item
    Comparison criterion: \underline{mean} squared error
    $E[(\hat\theta - \theta)^2]$
    \begin{itemize}
    \item The mean part can be approximated through replicates.
    \end{itemize}
  \end{itemize}
\end{frame}


\begin{frame}
  \frametitle{Uncertainty in Estimator}
  \begin{itemize}
  \item 
    Never just give a point estimate.
  \item
    Confidence interval: an interval estimator of the target with an
    associated confidence level that gives the probability with which
    the interval contain the target.
  \item
    Example: a 95\% confidence interval for the population mean
    $(L, U)$ means that this interval has probability 95\% covering
    the target.
  \item
    Construction
    \begin{itemize}
    \item Asymptotic results (theory that approximates well the
      reality when the sample size is large enough)
    \item Bootstrap (self-assisted).
    \end{itemize}
  \end{itemize}
\end{frame}


\begin{frame}
  \frametitle{Confidence Interval for Population Mean $\mu$}
  \begin{itemize}
  \item
    From theory (CLT, LLN, Slutsky)
    \[
      \frac{\bar X_n - \mu}{S_n/\sqrt{n}} \sim N(0, 1)
    \]
    where $\bar X_n$ is sample mean and $S_n^2$ is the sample variance.
  \item So
    \[
      \Pr\left(-z_{\alpha/2} < \frac{\sqrt{n}(\bar X_n - \mu)}{S_n}
        < z_{\alpha/2} \right)
      = 1 - \alpha
    \]
    where $z_{\alpha/2}$ is the upper $1 - \alpha/2$ quantile of
    $N(0, 1)$ (In R,  \texttt{qnorm(1 - alpha/2)} ).
    
  \item A $(1 - \alpha)$ confidence interval is
    \[
      \left(\bar X_n - z_{\alpha/2} \frac{ S_n }{\sqrt{n}}, \qquad
        \bar X_n + z_{\alpha/ 2} \frac{S_n} {\sqrt{n}}\right)
    \]
  \item Narrower for larger sample (more information)
  \end{itemize}
\end{frame}


\begin{frame}
  \frametitle{Bootstrap Confidence Interval (General)}
  \begin{itemize}
  \item
    One of the breakthroughs in Statistics (Efron, 1979).
  \item
    Widely applicable; no need for variance derivation/estimation; at the
    cost of intensive computation.
  \item A bootstrap copy of the estimator:
    \begin{itemize}
    \item Resample with replacement to form a bootstrap sample.
    \item Compute the estimate for the bootstrap sample.
    \end{itemize}
  \item Repeat the resampling many times ($B$) and obtain a large collection
    of bootstrap copies of the estimates
  \item Use the empirical quantiles of the collections to construct
    confidence interval for the target for this estimator.
  \item
    Theoretically justified for large $B$ and $n$.
   \end{itemize}
\end{frame}


\begin{frame}
  \frametitle{Summary}
  \begin{itemize}
  \item
    Statistical inference is a hide-and-seek game.
  \item
    Statisticians/data scientists are detectives.
  \item
    Different estimator (strategies) have different performance under
    different scenarios.
  \item
    An estimator needs to have a companion uncertainty measure.
  \item
    Confidence intervals: A valid confidence interval has actual
    coverage equal to the nominal level.
  \end{itemize}
\end{frame}
\end{document}