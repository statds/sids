\documentclass[leqno]{beamer}

\usetheme{Storrs}
\usecolortheme{lily}
\useinnertheme{rounded}

\usepackage{setspace, graphicx,hyperref}
\usepackage{amsfonts,amsthm,amsmath}
\usepackage{caption}
\usepackage{color}
\usepackage{multicol}
\usepackage{textpos}
\usepackage{tikz}
\usepackage[authoryear]{natbib}
\usepackage[linesnumbered,algoruled,boxed,lined]{algorithm2e}
\hypersetup{
	colorlinks,
	citecolor=blue}
\usepackage{multirow}
\usepackage{booktabs}
\usepackage{multicol}
\usepackage{soul} %% striketrhough texts
\usepackage{bm}

\newtheorem{proposition}{Proposition}

% \usetheme{Madrid}
% \usecolortheme{seahorse}

%% notations
\newcommand{\E}{\mathbb{E}\,}
\newcommand{\V}{\mathbb{V}}

\newcommand{\blue}[1]{\textcolor{blue}{#1}}
\newcommand{\red}[1]{\textcolor{red}{#1}}

\usepackage{graphicx}
% graphic path
\graphicspath{{./}{../manuscript/image/}}

\setbeamercovered{transparent}
\setbeamertemplate{itemize item}[circle]
\setbeamertemplate{enumerate item}[default]
\setbeamertemplate{section in toc}[default]
\setbeamertemplate{subsection in toc}[default]
\setbeamertemplate{caption}[numbered]

% \AtBeginSection[]{ 
% 	\begin{frame}<beamer>{Overview} 
% 	\tableofcontents[currentsection] 
% \end{frame}} 

\title[Estimation]{
  Hypothesis Testing: Decision under Uncertainty}
\subtitle[]{UConn-PCS 2021, Data Science}
\author[Jun Yan]{Jun Yan$^{1,2,3}$}
\institute[UConn]{
$^{1}$Department of Statistics, University of Connecticut\\
$^{2}$Center for Population Health, UConn Health\\
$^{3}$Center for Environmental Sciences and Engineering, University of Connecticut
}

\date{June 28, 2021 @ UConn}

\begin{document}

\begin{frame}[plain]
\titlepage
\end{frame}


\begin{frame}
  \frametitle{Hypothesis Testing}
  \begin{itemize}
  \item
    Examples
    \begin{itemize}
    \item A new drug is effective in treating a disease
    \item A commercial is more effective than another one on a website
      (A-B test)
    \item The increase in global mean temperature is attributable to
      human activities.
    \item The 30-day hospital readmission rate it higher among
      minorities.
    \end{itemize}
  \item
    Setup: a null hypothesis $H_0$; an alternative hypothesis $H_1$
    (the one you want to ``prove'' in scientific research).
  \item
    Consider this as a game: the nature hides the truth; we want to make a
    decision by selecting one of the hypotheses; we may make mistakes.
  \end{itemize}
\end{frame}


\begin{frame}
  \frametitle{Lady Tasting Tea}
  \begin{itemize}
  \item
    Popular science book: The Lady Tasting Tea: How Statistics
    Revolutionized Science in the Twentieth Century written by
    Dr. David Salsburg (2002), PhD in Statistics from UConn.
  \item
    A famous randomized experiment devised by
    \href{https://en.wikipedia.org/wiki/Ronald_Fisher}{Sir Ronald A. Fisher} and
    reported in his book The Design of Experiments (1935).
  \item
    A summer afternoon in 1920s at the Rothamsted Experimental
    Station.
  \item
    \href{https://en.wikipedia.org/wiki/Muriel_Bristol}{B. Muriel
      Bristol-Roach}, Ph.D., biologist at the Rothamsted Experimental
    Station.
  \item
    See \url{https://en.wikipedia.org/wiki/Lady_tasting_tea}.
  \item
    The lady got all 8 cups right (4 successes in the 4 selected
    cups), according to H. Fairfield Smith (UConn Stat Professor), who
    was there that afternoon.
  \end{itemize}
\end{frame}


\begin{frame}
  \frametitle{Test of Significance}
  \begin{itemize}
  \item
    Do you have significant evidence against the null hypothesis?
  \item
    Stochastic version of proof by contradiction.
  \item
    Idea:
    \begin{itemize}
    \item
      Start from the null hypothesis, compute the probability of
      observing as extreme as or more extreme than the observed
      statistic against the null hypothesis.
    \item
      If the probability is lower than some threshold, that’s support of
      significance.
    \end{itemize}
  \item
    Type I error: rejecting the null when it is true.
  \item
    Type II error: not rejecting the null when it is false.
  \end{itemize}
\end{frame}


\begin{frame}
  \frametitle{Properties of a Test}
  \begin{itemize}
  \item
    Testing statistic: a statistic whose distribution under $H_0$ is
    known and is sensitive to deviation from $H_0$.
  \item
    P-value: The probability of observing as extreme as or more
    extreme than the observed statistic against the null hypothesis.    
  \item
    Type I error rate: probability of rejecting the true $H_0$.
  \item
    Type II error rate: probability of accepting the false $H_0$.
  \item
    Power: probability of rejecting false $H_0$; 1 minus type II error rate.
  \item
    Power curve: Power as a function of deviation from $H_0$.
  \item
    The higher the better. 
  \end{itemize}
\end{frame}


\begin{frame}
  \frametitle{Summary}
  \begin{itemize}
  \item
    Hypothesis testing is also a hide-and-seek game.
  \item
    A decision may have types of errors.
  \item
    Under the $H_0$, a good test should maintain its size.
  \item
    Under the $H_1$, a good test should have substantial power.
  \item
    Test can be based on permutation or asymptotic results.
  \end{itemize}
\end{frame}
\end{document}