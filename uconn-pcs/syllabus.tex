\documentclass[twocolumn]{article}
\usepackage[hmargin={0.5in, 0.5in}, vmargin={0.8in, 0.8in}]{geometry}
\usepackage{booktabs}
\usepackage{hyperref}
\usepackage[anythingbreaks]{breakurl}

\newcommand{\pkg}[1]{{\normalfont\fontseries{b}\selectfont #1}}
\let\proglang=\textsf
\let\code=\texttt


\usepackage{enumitem}
\setlist{parsep=0pt, leftmargin=4mm, topsep=0pt, itemsep=8pt}
% \setlist{nolistsep}

\usepackage{fancyhdr}
\pagestyle{fancy}
\renewcommand{\headrulewidth}{0.4pt}
\renewcommand{\footrulewidth}{0.4pt}

\lhead{\sf UConn PCS: Data Science}
\rhead{summer 2021}
\lfoot{}
\rfoot{} %\url{http://www.stat.uconn.edu/~jyan/}}


\begin{document}

\title{UConn PCS: Data Science, Summer 2021}
\date{June 14, 2021}
\maketitle

\thispagestyle{fancy}


\begin{description}
\item[Instructors:] \hspace{0pt}

  Haim Bar: haim.bar@uconn.edu\\
  HaiYing Wang: haiying.wang@uconn.edu\\
  Jun Yan: jun.yan@uconn.edu

\item[HuskyCT:] Login with your NetID and password at
  \url{https://lms.uconn.edu}
  
\item[Lectures:] 
  MWF: 11:00am -- 1:30 pm ET @ HuskyCT

\item[Office Hours:]
  TBD (should not conflict with other program activities)
  % MWF: 2:30pm -- 3:30 pm ET @ HuskyCT


\item[Course Description:]
Data science is a fast developing science of extracting meaningful
information from massive data for better decision making. It is
interdisciplinary by nature, involving statistics, computing, and
domain knowledge. Important principles of data science will be
elaborated through interactive simulations, games and examples.


\item[Teaching Notes:] 
  Evolving pdf version in HuskyCT.
  
\item[Course Material:]

Topics include
\begin{enumerate}[noitemsep]
\item
  Introduction to R/RStudio, random number generation, data summaries
\item
  Exploratory data analysis, visualization
\item
  Paradoxes explained
\item
  At the infinity horizon (law of large number; central limit theorem)
\item
  Hide-and-seek: Estimating the unknowns
\item
  Sampling design, data collection
\item
  Correlation and regression
\item
  Hypothesis testing
\item
  A live project/capstone
\end{enumerate}

Announcements, homework assignment, and other course
information will be posted on HuskyCT.

\item[Computing:]
  Students are required to use \texttt{R} and \texttt{R Markdown}
  for homework assignments, exam, and project.

\item[Useful Resources:] Pick them up on the fly.
\begin{itemize}[noitemsep]
\item
A quick introduction to R programming language at
\url{https://cran.r-project.org/doc/manuals/r-release/R-intro.pdf}
\item
For an even shorter introduction see
\url{https://cran.r-project.org/doc/contrib/Torfs+Brauer-Short-R-Intro.pdf}
\item
Former UConn Stat PhD student Dr. Wenjie Wang's tutorial on R:
\url{https://github.com/wenjie2wang/2018-01-19-siam}
\item
Introduction to R Markdown by Garrett Grolemund:
\url{https://rmarkdown.rstudio.com/articles_intro.html}
\item
Hadley Wickham's tidyverse style guide on R:
\url{https://style.tidyverse.org}
\item
Hadley Wickham's Advanced R:
\url{https://bookdown.org/home/tags/advanced-r/}
\item
Write cool homework or project report with \texttt{bookdown} of Yihui Xie:
\url{https://bookdown.org/yihui/bookdown/}
\item
Happy git with R of Jenny Bryan:
\url{http://happygitwithr.com/}
\item
Build your own website with \texttt{blogdown}:
\url{https://bookdown.org/yihui/blogdown/}
\end{itemize}

\item[Grading:] Students will grade each other's assignment.
% Specific instructions about the procedure are provides in a separate file.

\begin{itemize}
\item
  Within one day after the deadline of an assignment, you will find your
randomly assigned classmate on HuskyCT.

\item
Email your solution to your assigned classmate, exactly as you
submitted it online. You have 1 day to provide your feedback. 

\item
You will submit your feedback on HuskyCT. For example, for Homework
1, there will be a follow-up assignment called "Homework 1 -
feedback". There, you will enter your numeric assessment using the
following scale:

1 = missing, or totally inadequate\\
2 = requires a major revision\\
3 = requires a minor revision\\
4 = complete and accurate solution

A template for the feedback is provided on HuskyCT.

\item
  If revisions are necessary, provide the necessary details in the
designated box.
\end{itemize}

Note that your assessment will NOT affect your partner's grade! The
purpose of this process is to learn from each other and develop good
programming and communication skills, so please be thorough and
honest, but also polite.

% \item[Notes:] 
%   \begin{itemize}
%   \item Pick up and sharpen computing skills on the fly. % on your own time.
%   \item Learn coding from the codes of experts. % Work on the problems in the textbook.
%   \end{itemize}

\end{description}

\end{document}

# Questions for PCS
+ do we need to give a grade?
+ any recommendation for breaks?
