\documentclass[11pt, twocolumn]{article}
\usepackage[hmargin={0.5in, 0.5in}, vmargin={0.8in, 0.8in}]{geometry}
\usepackage{booktabs}
\usepackage{hyperref}
\usepackage[anythingbreaks]{breakurl}

\newcommand{\pkg}[1]{{\normalfont\fontseries{b}\selectfont #1}}
\let\proglang=\textsf
\let\code=\texttt


\usepackage{enumitem}
\setlist{parsep=0pt, leftmargin=4mm, topsep=0pt, itemsep=8pt}
% \setlist{nolistsep}

\usepackage{fancyhdr}
\pagestyle{fancy}
\renewcommand{\headrulewidth}{0.4pt}
\renewcommand{\footrulewidth}{0.4pt}

\lhead{\sf UConn PCS: Data Science}
\rhead{summer 2021}
\lfoot{}
\rfoot{} %\url{http://www.stat.uconn.edu/~jyan/}}


\begin{document}

\title{UConn PCS: Data Science, Summer 2021}
\date{June 14, 2021}
\maketitle

\thispagestyle{fancy}


\begin{description}
\item[Instructors:]\hspace{0pt}

  Haim Bar: haim.bar@uconn.edu\\
  HaiYing Wang: haiying.wang@uconn.edu\\
  Jun Yan: jun.yan@uconn.edu

\item[Lectures:] 
  MWF: 11:00am -- 1:30 pm ET @ WebEx Meeting

\item[Office Hours:] 
  TBA


\item[Course Description:]

\item[Prerequisite:]
 

\item[Textbook:] 
  No textbook, but a set of teaching notes.

\item[Course Material:]

Topics include
\begin{enumerate}[noitemsep]
\item
  Introduction to R/RStudio
\item
  Paradoxes
\item
  Central Limit Theorems
\end{enumerate}

Announcements, lecture notes, homework assignment, and other course
information will be posted on HuskyCT (\url{lms.uconn.edu}). 
You are expected to read the sections of the textbook that will be covered.

\item[Computing:]
Students are required to use \texttt{RMarkdown}/\texttt{bookdown} and
\texttt{GitHub} for homework assignments, exam, and project.

% Other programming languages or software, such as SAS,
% Excel, C, Fortran, are \ink{0,0,1}{\emph{not\/}} allowed.

A quick introduction to R programming language at
\url{https://cran.r-project.org/doc/manuals/r-release/R-intro.pdf}

For an even shorter introduction see
\url{https://cran.r-project.org/doc/contrib/Torfs+Brauer-Short-R-Intro.pdf}

My former PhD student Dr. Wenjie Wang's tutorial on R:
\url{https://github.com/wenjie2wang/2018-01-19-siam}

Hadley Wickham's Advanced R:
\url{https://bookdown.org/home/tags/advanced-r/}
  
Write cool homework or project report with \texttt{bookdown} of Yihui Xie:
\url{https://bookdown.org/yihui/bookdown/}

% Learn \texttt{RMarkdown} at
% \url{rmarkdown.rstudio.com/}

Happy git with R of Jenny Bryan:
\url{http://happygitwithr.com/}

Build your own website with \texttt{blogdown}:
\url{https://bookdown.org/yihui/blogdown/}


\item[Grading:] Assessment of the subject will be based on 
the following four components with weights shown in parenthesis:

\begin{description}
\item[Homework (40\%)]
Collaborations are allowed for \emph{some (not all)}
homework assignments, but each group
should have at most 2 members. No collaborations between groups are
allowed. On the front page of solutions, the members of a group must be
clearly listed. Late homework will \emph{not} be accepted for \emph{any}
reason. 

\item[Exam (30\%)]
There will be one midterm take-home exam. The rule on collaborations for the
midterm exam is the same as for homework assignments.

\item[Project (30\%)]
Each group is expected to complete a class project on a topic of your choice
about statistical computing. There are many possibilities. For example, you
may review an important topic about statistical computation,
reproduce the results in a recent paper, solve a
real problem using comprehensive computation methods, investigate
properties of a computational algorithm/strategy, or compare several
computation methods on a variety of problems.

A final paper/report of your project using my template should be 6~pages in
length, excluding any appendices you wish to attach. In any case, the project
should present new work, not something you have done for another course.
If you use any reference, you must cite and credit your sources.
Outstanding report will be recommended for publication at the project showcases
of the Data Science Lab (\url{https://statds.org}).

% Your project
% will be shared with the class through the class website on HuskyCT.

\end{description}

% Grading scale will be close to (but may be slightly adjusted): 
% A (87--100), B (73--86), C (60--72), D (50--59), and F (0-50).
% Final grades will include $+/-$.

Grades for the course are assigned at the instructor's discretion.
% A \textbf{rough} guide:
% \begin{center}
%   \begin{tabular}{lllll}
%     A: 91--100\% & A$-$: 89--90\% & B$+$: 87--88\% & B: 81--86\%\\
%     B$-$: 79--80\%  & C$+$: 77--78\% & C: 71--76\% &  C$-$: 69--70\%\\
%     D$+$: 67--68\% &  D: 61--66\%  & D$-$: 59--60\% & F: $<$ 59\%
%   \end{tabular}
% \end{center}

\item[Notes:]\hspace{0pt}
%\small
  \begin{itemize}
  \item Pick up and sharpen computing skills on the fly. % on your own time.
  \item Learn coding from the codes of experts. % Work on the problems in the textbook.
  \item International students please consider English as part of your training.
  \item Academic integrity is seriously regarded and academic misconduct
    has severe consequences; students from different cultures beware.
  \item Learn some etiquettes in academia and practice them (e.g., how
    to email your professors).
  \end{itemize}

\end{description}

\end{document}
