\hypertarget{ch:estimation}{%
\chapter{Estimation: A hide and seek game with the nature}\label{ch:estimation}}

Consider this game setup. We have a random sample of size \(n\) from a normal
distribution \(N(\mu, 1)\), where \(\mu\) is unknown, and we want to estimate \(\mu\)
with this sample. An estimator is a quantity constructed from the observed
sample, that is, it is a statistic. What estimators can we construct? How do we
assess which is better?

\input{Rnw/est-hide-normal}

% Let's set up a true \(\mu\) that no one knows and generate a random sample of size
% \(n = 20\).

% \begin{Shaded}
% \begin{Highlighting}[]
% \FunctionTok{set.seed}\NormalTok{(}\DecValTok{123}\NormalTok{) }\CommentTok{\# each student can contribute a digit to make mu really unknown}
% \NormalTok{mu }\OtherTok{\textless{}{-}} \FunctionTok{runif}\NormalTok{(}\DecValTok{1}\NormalTok{, }\AttributeTok{min =}\SpecialCharTok{{-}} \DecValTok{10}\NormalTok{, }\AttributeTok{max =} \DecValTok{10}\NormalTok{)}
% \NormalTok{n }\OtherTok{\textless{}{-}} \DecValTok{20}
% \NormalTok{x }\OtherTok{\textless{}{-}} \FunctionTok{rnorm}\NormalTok{(n, }\AttributeTok{mean =}\NormalTok{ mu, }\AttributeTok{sd =} \DecValTok{1}\NormalTok{)}
% \FunctionTok{summary}\NormalTok{(x)}
% \end{Highlighting}
% \end{Shaded}

% \begin{verbatim}
%    Min. 1st Qu.  Median    Mean 3rd Qu.    Max. 
%  -5.976  -4.520  -3.877  -3.946  -3.046  -1.720 
% \end{verbatim}

% \textbf{Estimators}

% Let us consider a few candidate estimators

% \begin{itemize}
% \tightlist
% \item
%   Sample mean,
% \item
%   Sample median.
% \item
%   Midrange.
% \end{itemize}

% The three estimates based on the observed sample are

% \begin{Shaded}
% \begin{Highlighting}[]
% \NormalTok{(estimates }\OtherTok{\textless{}{-}} \FunctionTok{c}\NormalTok{(}\AttributeTok{mu1 =} \FunctionTok{mean}\NormalTok{(x), }\AttributeTok{mu2 =} \FunctionTok{median}\NormalTok{(x), }\AttributeTok{mu3 =} \FunctionTok{mean}\NormalTok{(}\FunctionTok{range}\NormalTok{(x))))}
% \end{Highlighting}
% \end{Shaded}

% \begin{verbatim}
%       mu1       mu2       mu3 
% -3.945533 -3.877437 -3.847917 
% \end{verbatim}

% \textbf{Mean Squared Error}
% To assess which estimator is better, we compare them with the true value of
% \(\mu\). The squared error of an estimator \(\hat\mu\) of \(\mu\) is
% \((\hat\mu - \mu)^2\).
% For the three estimates based on the given sample, we have:

% \begin{Shaded}
% \begin{Highlighting}[]
% \NormalTok{(estimates }\SpecialCharTok{{-}}\NormalTok{ mu)}\SpecialCharTok{\^{}}\DecValTok{2}
% \end{Highlighting}
% \end{Shaded}

% \begin{verbatim}
%        mu1        mu2        mu3 
% 0.09175876 0.13765041 0.16042665 
% \end{verbatim}

% Note that this comparison is for only the given sample. A good estimator may by
% chance behaves worse than a bad estimator. A real assessment of the quality of
% the estimator should be based on a large number of replicates, where we can
% check which estimator performs the best on average. To do so, we need to play
% this game repeatedly, collect the squared errors of the estimators, and compare
% their means, that is, mean squared error.

% Now the game becomes a racing game. We do it through a simulation study.
% Let us construct a function to do replicate of such game.

% \begin{Shaded}
% \begin{Highlighting}[]
% \NormalTok{do1rep }\OtherTok{\textless{}{-}} \ControlFlowTok{function}\NormalTok{(n, mu) \{}
%     \DocumentationTok{\#\# generate data}
% \NormalTok{    x }\OtherTok{\textless{}{-}} \FunctionTok{rnorm}\NormalTok{(n, }\AttributeTok{mean =}\NormalTok{ mu, }\AttributeTok{sd =} \DecValTok{1}\NormalTok{)}
%     \DocumentationTok{\#\# collect estimates}
% \NormalTok{    est }\OtherTok{\textless{}{-}}  \FunctionTok{c}\NormalTok{(}\AttributeTok{mu1 =} \FunctionTok{mean}\NormalTok{(x), }\AttributeTok{mu2 =} \FunctionTok{median}\NormalTok{(x), }\AttributeTok{mu3 =} \FunctionTok{mean}\NormalTok{(}\FunctionTok{range}\NormalTok{(x)))}
%     \DocumentationTok{\#\# return squared error}
%     \FunctionTok{return}\NormalTok{((est }\SpecialCharTok{{-}}\NormalTok{ mu)}\SpecialCharTok{\^{}}\DecValTok{2}\NormalTok{)}
% \NormalTok{\}}
% \end{Highlighting}
% \end{Shaded}

% Each time we call this function, the experiment is done and the squared errors
% of the three estimators are returned.

% \begin{Shaded}
% \begin{Highlighting}[]
% \FunctionTok{do1rep}\NormalTok{(n, mu)}
% \end{Highlighting}
% \end{Shaded}

% \begin{verbatim}
%        mu1        mu2        mu3 
% 0.17323711 0.07180305 1.16285198 
% \end{verbatim}

% \begin{Shaded}
% \begin{Highlighting}[]
% \FunctionTok{do1rep}\NormalTok{(n, mu)}
% \end{Highlighting}
% \end{Shaded}

% \begin{verbatim}
%        mu1        mu2        mu3 
% 0.02543683 0.02323603 0.03439818 
% \end{verbatim}

% Which estimator is winning the game? Let's repeat the experiment 1000 times and
% compare the MSE.

% \begin{Shaded}
% \begin{Highlighting}[]
% \NormalTok{nrep }\OtherTok{\textless{}{-}} \DecValTok{1000}
% \NormalTok{sim }\OtherTok{\textless{}{-}} \FunctionTok{replicate}\NormalTok{(nrep, }\FunctionTok{do1rep}\NormalTok{(n, mu))}
% \FunctionTok{rowMeans}\NormalTok{(sim)}
% \end{Highlighting}
% \end{Shaded}

% \begin{verbatim}
%        mu1        mu2        mu3 
% 0.05115881 0.07603630 0.13703344 
% \end{verbatim}

% The first estimator, the sample mean, is a clear winner!

% The area of a circle with radiur \(r\) is

% \begin{align}
%     R = \pi r^2.
% \label{eq:area}
% \end{align}

% \begin{align*}
% \mathrm{MSE}(\hat\mu) = (\hat\mu - \mu)^2
% \end{align*}

