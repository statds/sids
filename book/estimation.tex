\hypertarget{ch:estimation}{%
\chapter{Estimation: A hide-and-seek game with the nature}\label{ch:estimation}}


\section{Introduction}
Consider this game setup. We have a random sample of size \(n\) from a normal
distribution \(N(\mu, 1)\), where \(\mu\) is unknown, and we want to estimate \(\mu\)
with this sample. An estimator is a quantity constructed from the observed
sample, that is, it is a statistic. What estimators can we construct? How do we
assess which is better?

\input{Rnw/est-hide-normal}

\section{Seeking strategies}

To play the hide-and-seek game better, we need to select a good strategy. From
the examples in the last section, it seems that which strategy is better depends
on the underlying truth, which is unknown.


Let's start with an example.
\begin{illustration}[Likelihood in guessing jars]
  % \indexExSix{Computing ranks in the presence of ties}
\label{example:has-jars}
There are 9 jars. The first jar has 9 red and 1 green balls; the second jar has
8 red and 2 green balls; $\cdots\cdots$; the 9th jar has 1 red and 9 green
balls. Now, one jar is randomly selected, from which one ball is randomly
selected, and the ball is red. Which jar was the one that was selected?

There are nine possibilities. If the selected jar were the first one, the
probability that a randomly selected ball is red is 9/10. If the selected jar
were the second one, the probability that a randomly selected ball is red is
8/10. Continue until the 9th jar. With this calculation, which jar should we
guess?

The first one!
\end{illustration}

The principle we use here is the so-called maximum likelihood estimation. The
unknown true has 9 possibilities. We choose the one that gives the highest
likelihood of observing a randomly selected ball being red.


% The area of a circle with radiur \(r\) is

% \begin{align}
%     R = \pi r^2.
% \label{eq:area}
% \end{align}

% \begin{align*}
% \mathrm{MSE}(\hat\mu) = (\hat\mu - \mu)^2
% \end{align*}

