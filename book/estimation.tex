\hypertarget{ch:estimation}{%
\chapter{Estimation: A hide-and-seek game with the nature}\label{ch:estimation}}


\section{Introduction}
Consider this game setup. We have a random sample of size \(n\) from a normal
distribution \(N(\mu, 1)\), where \(\mu\) is unknown, and we want to estimate \(\mu\)
with this sample. An estimator is a quantity constructed from the observed
sample, that is, it is a statistic. What estimators can we construct? How do we
assess which is better?

Let's set up a true $\mu$ that no one knows and generate a random sample of size
$n = 20$.
\begin{knitrout}
\definecolor{shadecolor}{rgb}{0.969, 0.969, 0.969}\color{fgcolor}\begin{kframe}
\begin{alltt}
\hlkwd{set.seed}\hlstd{(}\hlnum{123}\hlstd{)} \hlcom{# each student can contribute a digit to make mu really unknown}
\hlstd{mu} \hlkwb{<-} \hlkwd{runif}\hlstd{(}\hlnum{1}\hlstd{,} \hlkwc{min} \hlstd{=}\hlopt{-} \hlnum{10}\hlstd{,} \hlkwc{max} \hlstd{=} \hlnum{10}\hlstd{)}
\hlstd{n} \hlkwb{<-} \hlnum{20}
\hlstd{x} \hlkwb{<-} \hlkwd{rnorm}\hlstd{(n,} \hlkwc{mean} \hlstd{= mu,} \hlkwc{sd} \hlstd{=} \hlnum{1}\hlstd{)}
\hlkwd{summary}\hlstd{(x)}
\end{alltt}
\begin{verbatim}
##    Min. 1st Qu.  Median    Mean 3rd Qu.    Max. 
##  -5.976  -4.520  -3.877  -3.946  -3.046  -1.720
\end{verbatim}
\end{kframe}
\end{knitrout}


\paragraph{Estimators}

Let us consider a few candidate estimators

\begin{itemize}
\item Sample mean
\item Sample median
\item Midrange
\end{itemize}

The three estimates based on the observed sample are
\begin{knitrout}
\definecolor{shadecolor}{rgb}{0.969, 0.969, 0.969}\color{fgcolor}\begin{kframe}
\begin{alltt}
\hlstd{(estimates} \hlkwb{<-} \hlkwd{c}\hlstd{(}\hlkwc{mu1} \hlstd{=} \hlkwd{mean}\hlstd{(x),} \hlkwc{mu2} \hlstd{=} \hlkwd{median}\hlstd{(x),} \hlkwc{mu3} \hlstd{=} \hlkwd{mean}\hlstd{(}\hlkwd{range}\hlstd{(x))))}
\end{alltt}
\begin{verbatim}
##       mu1       mu2       mu3 
## -3.945533 -3.877437 -3.847917
\end{verbatim}
\end{kframe}
\end{knitrout}


\paragraph{Mean Squared Error}

To assess which estimator is better, we compare them with the true value of
$\mu$. The squared error of an estimator $\hat\mu$ of $\mu$ is
$(\hat\mu - \mu)^2$.
For the three estimates based on the given sample, we have:
\begin{knitrout}
\definecolor{shadecolor}{rgb}{0.969, 0.969, 0.969}\color{fgcolor}\begin{kframe}
\begin{alltt}
\hlstd{(estimates} \hlopt{-} \hlstd{mu)}\hlopt{^}\hlnum{2}
\end{alltt}
\begin{verbatim}
##        mu1        mu2        mu3 
## 0.09175876 0.13765041 0.16042665
\end{verbatim}
\end{kframe}
\end{knitrout}


Note that this comparison is for only the given sample. A good estimator may by
chance behaves worse than a bad estimator. A real assessment of the quality of
the estimator should be based on a large number of replicates, where we can
check which estimator performs the best on average. To do so, we need to play
this game repeatedly, collect the squared errors of the estimators, and compare
their means, that is, mean squared error.


\begin{illustration}[Hide-and-seek with a normal population]
Now the game becomes a racing game. We do it through a simulation study.
Let us construct a function to do replicate of such game.
\begin{knitrout}
\definecolor{shadecolor}{rgb}{0.969, 0.969, 0.969}\color{fgcolor}\begin{kframe}
\begin{alltt}
\hlstd{do1rep} \hlkwb{<-} \hlkwa{function}\hlstd{(}\hlkwc{n}\hlstd{,} \hlkwc{mu}\hlstd{) \{}
    \hlcom{## generate data}
    \hlstd{x} \hlkwb{<-} \hlkwd{rnorm}\hlstd{(n,} \hlkwc{mean} \hlstd{= mu,} \hlkwc{sd} \hlstd{=} \hlnum{1}\hlstd{)}
    \hlcom{## collect estimates}
    \hlstd{est} \hlkwb{<-}  \hlkwd{c}\hlstd{(}\hlkwc{mu1} \hlstd{=} \hlkwd{mean}\hlstd{(x),} \hlkwc{mu2} \hlstd{=} \hlkwd{median}\hlstd{(x),} \hlkwc{mu3} \hlstd{=} \hlkwd{mean}\hlstd{(}\hlkwd{range}\hlstd{(x)))}
    \hlcom{## return squared error}
    \hlkwd{return}\hlstd{((est} \hlopt{-} \hlstd{mu)}\hlopt{^}\hlnum{2}\hlstd{)}
\hlstd{\}}
\end{alltt}
\end{kframe}
\end{knitrout}

Each time we call this function, the experiment is done and the squared errors
of the three estimators are returned.
\begin{knitrout}
\definecolor{shadecolor}{rgb}{0.969, 0.969, 0.969}\color{fgcolor}\begin{kframe}
\begin{alltt}
\hlkwd{do1rep}\hlstd{(n, mu)}
\end{alltt}
\begin{verbatim}
##        mu1        mu2        mu3 
## 0.17323711 0.07180305 1.16285198
\end{verbatim}
\begin{alltt}
\hlkwd{do1rep}\hlstd{(n, mu)}
\end{alltt}
\begin{verbatim}
##        mu1        mu2        mu3 
## 0.02543683 0.02323603 0.03439818
\end{verbatim}
\end{kframe}
\end{knitrout}



Which estimator is winning the game? Let's repeat the experiment 1000 times and
compare the MSE.
\begin{knitrout}
\definecolor{shadecolor}{rgb}{0.969, 0.969, 0.969}\color{fgcolor}\begin{kframe}
\begin{alltt}
\hlstd{nrep} \hlkwb{<-} \hlnum{1000}
\hlstd{sim} \hlkwb{<-} \hlkwd{replicate}\hlstd{(nrep,} \hlkwd{do1rep}\hlstd{(n, mu))}
\hlkwd{rowMeans}\hlstd{(sim)}
\end{alltt}
\begin{verbatim}
##        mu1        mu2        mu3 
## 0.05115881 0.07603630 0.13703344
\end{verbatim}
\end{kframe}
\end{knitrout}


The first estimator, the sample mean, is a clear winner!
\end{illustration}

\begin{illustration}[Hide-and-seek with a Cauchy population]
  % \indexExSix{Computing ranks in the presence of ties}
\label{example:has-cauchy}
Now let's change the distribution of the population from normal to Cauchy.
\begin{knitrout}
\definecolor{shadecolor}{rgb}{0.969, 0.969, 0.969}\color{fgcolor}\begin{kframe}
\begin{alltt}
\hlkwd{set.seed}\hlstd{(}\hlnum{1979}\hlstd{)}
\hlstd{mu} \hlkwb{<-} \hlkwd{runif}\hlstd{(}\hlnum{1}\hlstd{,} \hlkwc{min} \hlstd{=}\hlopt{-} \hlnum{10}\hlstd{,} \hlkwc{max} \hlstd{=} \hlnum{10}\hlstd{)}
\hlstd{x} \hlkwb{<-} \hlkwd{rcauchy}\hlstd{(n,} \hlkwc{location} \hlstd{= mu)}
\hlkwd{summary}\hlstd{(x)}
\end{alltt}
\begin{verbatim}
##    Min. 1st Qu.  Median    Mean 3rd Qu.    Max. 
##   3.871   7.999   8.884  10.031   9.821  25.848
\end{verbatim}
\begin{alltt}
\hlstd{do1rep} \hlkwb{<-} \hlkwa{function}\hlstd{(}\hlkwc{n}\hlstd{,} \hlkwc{mu}\hlstd{) \{}
    \hlcom{## generate data}
    \hlstd{x} \hlkwb{<-} \hlkwd{rcauchy}\hlstd{(n,} \hlkwc{location} \hlstd{= mu)}
    \hlcom{## collect estimates}
    \hlstd{est} \hlkwb{<-}  \hlkwd{c}\hlstd{(}\hlkwc{mu1} \hlstd{=} \hlkwd{mean}\hlstd{(x),} \hlkwc{mu2} \hlstd{=} \hlkwd{median}\hlstd{(x),} \hlkwc{mu3} \hlstd{=} \hlkwd{mean}\hlstd{(}\hlkwd{range}\hlstd{(x)))}
    \hlcom{## return squared error}
    \hlkwd{return}\hlstd{((est} \hlopt{-} \hlstd{mu)}\hlopt{^}\hlnum{2}\hlstd{)}
\hlstd{\}}

\hlstd{sim} \hlkwb{<-} \hlkwd{replicate}\hlstd{(nrep,} \hlkwd{do1rep}\hlstd{(n, mu))}
\hlkwd{rowMeans}\hlstd{(sim)}
\end{alltt}
\begin{verbatim}
##          mu1          mu2          mu3 
## 3.133503e+04 1.430156e-01 3.128879e+06
\end{verbatim}
\end{kframe}
\end{knitrout}
Which estimator is the winner this time?
\end{illustration}


\begin{illustration}[Hide-and-seek with a uniform population]
  % \indexExSix{Computing ranks in the presence of ties}
\label{example:has-unif}
Now let's change the distribution of the population to one with light tails.
\begin{knitrout}
\definecolor{shadecolor}{rgb}{0.969, 0.969, 0.969}\color{fgcolor}\begin{kframe}
\begin{alltt}
\hlkwd{set.seed}\hlstd{(}\hlnum{1975}\hlstd{)}
\hlstd{mu} \hlkwb{<-} \hlkwd{runif}\hlstd{(}\hlnum{1}\hlstd{,} \hlkwc{min} \hlstd{=}\hlopt{-} \hlnum{10}\hlstd{,} \hlkwc{max} \hlstd{=} \hlnum{10}\hlstd{)}
\hlstd{x} \hlkwb{<-} \hlkwd{runif}\hlstd{(n, mu} \hlopt{-} \hlnum{1}\hlstd{, mu} \hlopt{+} \hlnum{1}\hlstd{)}
\hlkwd{summary}\hlstd{(x)}
\end{alltt}
\begin{verbatim}
##    Min. 1st Qu.  Median    Mean 3rd Qu.    Max. 
##  0.3267  0.8011  1.4551  1.2868  1.6794  2.1342
\end{verbatim}
\begin{alltt}
\hlstd{do1rep} \hlkwb{<-} \hlkwa{function}\hlstd{(}\hlkwc{n}\hlstd{,} \hlkwc{mu}\hlstd{) \{}
    \hlcom{## generate data}
    \hlstd{x} \hlkwb{<-} \hlkwd{runif}\hlstd{(n, mu} \hlopt{-} \hlnum{1}\hlstd{, mu} \hlopt{+} \hlnum{1}\hlstd{)}
    \hlcom{## collect estimates}
    \hlstd{est} \hlkwb{<-}  \hlkwd{c}\hlstd{(}\hlkwc{mu1} \hlstd{=} \hlkwd{mean}\hlstd{(x),} \hlkwc{mu2} \hlstd{=} \hlkwd{median}\hlstd{(x),} \hlkwc{mu3} \hlstd{=} \hlkwd{mean}\hlstd{(}\hlkwd{range}\hlstd{(x)))}
    \hlcom{## return squared error}
    \hlkwd{return}\hlstd{((est} \hlopt{-} \hlstd{mu)}\hlopt{^}\hlnum{2}\hlstd{)}
\hlstd{\}}

\hlstd{sim} \hlkwb{<-} \hlkwd{replicate}\hlstd{(nrep,} \hlkwd{do1rep}\hlstd{(n, mu))}
\hlkwd{rowMeans}\hlstd{(sim)}
\end{alltt}
\begin{verbatim}
##         mu1         mu2         mu3 
## 0.015830990 0.041249277 0.004328254
\end{verbatim}
\end{kframe}
\end{knitrout}
Which estimator is the winner this time?
\end{illustration}


Can we find an estimator that performs the best in all scenarios?

Now that the performance depends on the true population which is
unknown, can we make some good guesses and choose one? This is known
as adaptive estimation.


The location parameter of a distribution controls the location of the
distribution.


\section{Seeking strategies}

To play the hide-and-seek game better, we need to select a good strategy. From
the examples in the last section, it seems that which strategy is better depends
on the underlying truth, which is unknown.


Let's start with an example.
\begin{illustration}[Likelihood in guessing jars]
  % \indexExSix{Computing ranks in the presence of ties}
\label{example:has-jars}
There are 9 jars. The first jar has 9 red and 1 green balls; the second jar has
8 red and 2 green balls; $\cdots\cdots$; the 9th jar has 1 red and 9 green
balls. Now, one jar is randomly selected, from which one ball is randomly
selected, and the ball is red. Which jar was the one that was selected?

There are nine possibilities. If the selected jar were the first one, the
probability that a randomly selected ball is red is 9/10. If the selected jar
were the second one, the probability that a randomly selected ball is red is
8/10. Continue until the 9th jar. With this calculation, which jar should we
guess?

The first one!
\end{illustration}

The principle we use here is the so-called maximum likelihood estimation. The
unknown true has 9 possibilities. We choose the one that gives the highest
likelihood of observing a randomly selected ball being red.

\section{Uncertainty}

Now let's play a game to appreciate the importance of uncertainty in
estimation. We will use the asymptotic theory to construct the 95\%
confidence intervals and check their actural coverage rates.

\begin{illustration}
Let's generate some $\theta$ that we do not know.
\begin{knitrout}
\definecolor{shadecolor}{rgb}{0.969, 0.969, 0.969}\color{fgcolor}\begin{kframe}
\begin{alltt}
\hlkwd{set.seed}\hlstd{(}\hlnum{20210623}\hlstd{)}
\hlstd{theta}  \hlkwb{<-} \hlkwd{runif}\hlstd{(}\hlnum{1}\hlstd{,} \hlnum{1}\hlstd{,} \hlnum{10}\hlstd{)}
\hlstd{n} \hlkwb{<-} \hlnum{50}
\hlstd{x} \hlkwb{<-} \hlkwd{rgamma}\hlstd{(n,} \hlkwc{shape} \hlstd{= theta}\hlopt{/}\hlnum{2}\hlstd{,} \hlkwc{scale} \hlstd{=} \hlnum{2}\hlstd{)}
\hlstd{ciclt} \hlkwb{<-} \hlkwa{function}\hlstd{(}\hlkwc{x}\hlstd{,} \hlkwc{alpha} \hlstd{=} \hlnum{.05}\hlstd{) \{}
    \hlstd{xbar} \hlkwb{<-} \hlkwd{mean}\hlstd{(x)}
    \hlstd{ss} \hlkwb{<-} \hlkwd{sd}\hlstd{(x)}
    \hlstd{z} \hlkwb{<-} \hlkwd{qnorm}\hlstd{(}\hlnum{1} \hlopt{-} \hlstd{alpha} \hlopt{/} \hlnum{2}\hlstd{)}
    \hlstd{dd} \hlkwb{<-} \hlstd{z} \hlopt{*} \hlstd{ss} \hlopt{/} \hlkwd{sqrt}\hlstd{(}\hlkwd{length}\hlstd{(x))}
    \hlkwd{c}\hlstd{(xbar} \hlopt{-} \hlstd{dd, xbar} \hlopt{+} \hlstd{dd)}
\hlstd{\}}
\hlkwd{ciclt}\hlstd{(x)}
\end{alltt}
\begin{verbatim}
## [1] 2.089077 3.249442
\end{verbatim}
\begin{alltt}
\hlstd{x} \hlkwb{<-} \hlkwd{rgamma}\hlstd{(n,} \hlkwc{shape} \hlstd{= theta}\hlopt{/}\hlnum{2}\hlstd{,} \hlkwc{scale} \hlstd{=} \hlnum{2}\hlstd{)}
\hlkwd{ciclt}\hlstd{(x)}
\end{alltt}
\begin{verbatim}
## [1] 2.130422 3.616720
\end{verbatim}
\begin{alltt}
\hlstd{x} \hlkwb{<-} \hlkwd{rgamma}\hlstd{(n,} \hlkwc{shape} \hlstd{= theta}\hlopt{/}\hlnum{2}\hlstd{,} \hlkwc{scale} \hlstd{=} \hlnum{2}\hlstd{)}
\hlkwd{ciclt}\hlstd{(x)}
\end{alltt}
\begin{verbatim}
## [1] 1.985616 3.233137
\end{verbatim}
\end{kframe}
\end{knitrout}

Does a 95\% onfidence interval constructed this way really give 95\%
probability of covering the truth? By theory, it should when the
sample size is large. Is $n = 50$ large enough for this to be good?
We can check the actual coverage in a simulation study.
\begin{knitrout}
\definecolor{shadecolor}{rgb}{0.969, 0.969, 0.969}\color{fgcolor}\begin{kframe}
\begin{alltt}
\hlstd{nrep} \hlkwb{<-} \hlnum{1000}
\hlstd{do1rep} \hlkwb{<-} \hlkwa{function}\hlstd{(}\hlkwc{n}\hlstd{,} \hlkwc{theta}\hlstd{,} \hlkwc{alpha} \hlstd{=} \hlnum{.05}\hlstd{) \{}
    \hlcom{## generate data}
    \hlstd{x} \hlkwb{<-} \hlkwd{rgamma}\hlstd{(n,} \hlkwc{shape} \hlstd{= theta} \hlopt{/} \hlnum{2}\hlstd{,} \hlkwc{scale} \hlstd{=} \hlnum{2}\hlstd{)}
    \hlcom{## return the confidence interval}
    \hlkwd{ciclt}\hlstd{(x, alpha)}
\hlstd{\}}

\hlstd{sim} \hlkwb{<-} \hlkwd{replicate}\hlstd{(nrep,} \hlkwd{do1rep}\hlstd{(n, theta))}
\hlkwd{mean}\hlstd{(sim[}\hlnum{1}\hlstd{, ]} \hlopt{<} \hlstd{theta} \hlopt{&} \hlstd{sim[}\hlnum{2}\hlstd{,]} \hlopt{>} \hlstd{theta)}
\end{alltt}
\begin{verbatim}
## [1] 0.945
\end{verbatim}
\begin{alltt}
\hlcom{## let's try a smaller sample size}
\hlstd{sim} \hlkwb{<-} \hlkwd{replicate}\hlstd{(nrep,} \hlkwd{do1rep}\hlstd{(}\hlnum{20}\hlstd{, theta))}
\hlkwd{mean}\hlstd{(sim[}\hlnum{1}\hlstd{, ]} \hlopt{<} \hlstd{theta} \hlopt{&} \hlstd{sim[}\hlnum{2}\hlstd{,]} \hlopt{>} \hlstd{theta)}
\end{alltt}
\begin{verbatim}
## [1] 0.895
\end{verbatim}
\end{kframe}
\end{knitrout}
The simulation shows that the confidence intervals constructed from
the large sample theory works well for sample size $n = 50$ but not so
well for $n = 20$.
\end{illustration}


Here we construct bootstrap confidence interval for estimating the
population median of a gamma distribution.
\begin{knitrout}
\definecolor{shadecolor}{rgb}{0.969, 0.969, 0.969}\color{fgcolor}\begin{kframe}
\begin{alltt}
\hlkwd{set.seed}\hlstd{(}\hlnum{2021}\hlstd{)}
\hlstd{theta} \hlkwb{<-} \hlkwd{runif}\hlstd{(}\hlnum{1}\hlstd{,} \hlnum{1}\hlstd{,} \hlnum{10}\hlstd{)}
\hlstd{myBootCI} \hlkwb{<-} \hlkwa{function}\hlstd{(}\hlkwc{x}\hlstd{,} \hlkwc{alpha} \hlstd{=} \hlnum{.05}\hlstd{,} \hlkwc{B} \hlstd{=} \hlnum{400}\hlstd{,} \hlkwc{fun} \hlstd{= mean) \{}
    \hlcom{## mb <- replicate(B, mean(sample(x, size = length(x), replace = TRUE)))}
    \hlstd{mb} \hlkwb{<-} \hlkwd{rep}\hlstd{(}\hlnum{0}\hlstd{, B)}
    \hlkwa{for} \hlstd{(i} \hlkwa{in} \hlnum{1}\hlopt{:}\hlstd{B) \{}
        \hlstd{xb} \hlkwb{<-} \hlkwd{sample}\hlstd{(x,} \hlkwc{size} \hlstd{=} \hlkwd{length}\hlstd{(x),} \hlkwc{replace} \hlstd{=} \hlnum{TRUE}\hlstd{)}
        \hlstd{mb[i]} \hlkwb{<-} \hlkwd{fun}\hlstd{(xb)}
    \hlstd{\}}
    \hlkwd{quantile}\hlstd{(mb,} \hlkwd{c}\hlstd{(alpha} \hlopt{/} \hlnum{2}\hlstd{,} \hlnum{1} \hlopt{-} \hlstd{alpha} \hlopt{/} \hlnum{2}\hlstd{))}
\hlstd{\}}

\hlstd{n} \hlkwb{<-} \hlnum{50}
\hlstd{x} \hlkwb{<-} \hlkwd{rgamma}\hlstd{(n,} \hlkwc{shape} \hlstd{= theta} \hlopt{/} \hlnum{2}\hlstd{,} \hlkwc{scale} \hlstd{=} \hlnum{2}\hlstd{)}
\hlkwd{myBootCI}\hlstd{(x,} \hlkwc{fun} \hlstd{= mean)}
\end{alltt}
\begin{verbatim}
##     2.5%    97.5% 
## 3.603285 5.368040
\end{verbatim}
\begin{alltt}
\hlstd{nrep} \hlkwb{<-} \hlnum{1000}
\hlstd{do1rep} \hlkwb{<-} \hlkwa{function}\hlstd{(}\hlkwc{n}\hlstd{,} \hlkwc{theta}\hlstd{,} \hlkwc{alpha} \hlstd{=} \hlnum{.05}\hlstd{,} \hlkwc{B} \hlstd{=} \hlnum{400}\hlstd{,} \hlkwc{fun} \hlstd{= mean) \{}
    \hlcom{## generate data}
    \hlstd{x} \hlkwb{<-} \hlkwd{rgamma}\hlstd{(n,} \hlkwc{shape} \hlstd{= theta} \hlopt{/} \hlnum{2}\hlstd{,} \hlkwc{scale} \hlstd{=} \hlnum{2}\hlstd{)}
    \hlcom{## return the confidence interval}
    \hlkwd{myBootCI}\hlstd{(x, alpha, B, fun)}
\hlstd{\}}

\hlstd{sim} \hlkwb{<-} \hlkwd{replicate}\hlstd{(nrep,} \hlkwd{do1rep}\hlstd{(n, theta,} \hlkwc{fun} \hlstd{= mean))}
\hlkwd{mean}\hlstd{(sim[}\hlnum{1}\hlstd{, ]} \hlopt{<} \hlstd{theta} \hlopt{&} \hlstd{sim[}\hlnum{2}\hlstd{,]} \hlopt{>} \hlstd{theta)}
\end{alltt}
\begin{verbatim}
## [1] 0.933
\end{verbatim}
\begin{alltt}
\hlcom{## let's try a smaller sample size}
\hlstd{sim} \hlkwb{<-} \hlkwd{replicate}\hlstd{(nrep,} \hlkwd{do1rep}\hlstd{(}\hlnum{20}\hlstd{, theta))}
\hlkwd{mean}\hlstd{(sim[}\hlnum{1}\hlstd{, ]} \hlopt{<} \hlstd{theta} \hlopt{&} \hlstd{sim[}\hlnum{2}\hlstd{,]} \hlopt{>} \hlstd{theta)}
\end{alltt}
\begin{verbatim}
## [1] 0.899
\end{verbatim}
\end{kframe}
\end{knitrout}

The same code could be used to estimate other target, say the
population median.
\begin{knitrout}
\definecolor{shadecolor}{rgb}{0.969, 0.969, 0.969}\color{fgcolor}\begin{kframe}
\begin{alltt}
\hlcom{## Let's estimate the population median with sample median}
\hlstd{sim} \hlkwb{<-} \hlkwd{replicate}\hlstd{(nrep,} \hlkwd{do1rep}\hlstd{(}\hlnum{50}\hlstd{, theta,} \hlkwc{fun} \hlstd{= median))}
\hlstd{pmed} \hlkwb{<-} \hlkwd{qgamma}\hlstd{(}\hlnum{0.5}\hlstd{,} \hlkwc{shape} \hlstd{= theta} \hlopt{/} \hlnum{2}\hlstd{,} \hlkwc{scale} \hlstd{=} \hlnum{2}\hlstd{)}
\hlkwd{mean}\hlstd{(sim[}\hlnum{1}\hlstd{, ]} \hlopt{<} \hlstd{pmed} \hlopt{&} \hlstd{sim[}\hlnum{2}\hlstd{,]} \hlopt{>} \hlstd{pmed)}
\end{alltt}
\begin{verbatim}
## [1] 0.947
\end{verbatim}
\end{kframe}
\end{knitrout}


