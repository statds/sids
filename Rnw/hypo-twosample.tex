Consider testing whether two sample have the same mean.

\begin{knitrout}
\definecolor{shadecolor}{rgb}{0.969, 0.969, 0.969}\color{fgcolor}\begin{kframe}
\begin{alltt}
\hlkwd{set.seed}\hlstd{(}\hlnum{20210628}\hlstd{)}
\hlstd{delta} \hlkwb{<-} \hlnum{0}
\hlstd{n1} \hlkwb{<-} \hlstd{n2} \hlkwb{<-} \hlstd{n} \hlkwb{<-} \hlnum{30}
\hlstd{x1} \hlkwb{<-} \hlkwd{rnorm}\hlstd{(n)}
\hlstd{x2} \hlkwb{<-} \hlkwd{rnorm}\hlstd{(n2)} \hlopt{+} \hlstd{delta}
\hlkwd{t.test}\hlstd{(x1, x2)}
\end{alltt}
\begin{verbatim}
## 
## 	Welch Two Sample t-test
## 
## data:  x1 and x2
## t = 0.6515, df = 57.364, p-value = 0.5173
## alternative hypothesis: true difference in means is not equal to 0
## 95 percent confidence interval:
##  -0.3095775  0.6082215
## sample estimates:
##  mean of x  mean of y 
## 0.24627636 0.09695435
\end{verbatim}
\begin{alltt}
\hlcom{## understand the output}
\hlkwd{wilcox.test}\hlstd{(x1, x2)} \hlcom{# rank-based}
\end{alltt}
\begin{verbatim}
## 
## 	Wilcoxon rank sum exact test
## 
## data:  x1 and x2
## W = 488, p-value = 0.5819
## alternative hypothesis: true location shift is not equal to 0
\end{verbatim}
\end{kframe}
\end{knitrout}

Let's investigate the properties of the two tests. The validity of a
test is affected by the sample size $n$, the data-generating
distribution, the deviation $\delta$ from the null.
\begin{knitrout}
\definecolor{shadecolor}{rgb}{0.969, 0.969, 0.969}\color{fgcolor}\begin{kframe}
\begin{alltt}
\hlstd{do1rep} \hlkwb{<-} \hlkwa{function}\hlstd{(}\hlkwc{n}\hlstd{,} \hlkwc{datagen}\hlstd{,} \hlkwc{delta} \hlstd{=} \hlnum{0}\hlstd{) \{}
    \hlstd{x1} \hlkwb{<-} \hlkwd{datagen}\hlstd{(n)}
    \hlstd{x2} \hlkwb{<-} \hlkwd{datagen}\hlstd{(n)} \hlopt{+} \hlstd{delta}
    \hlstd{p1} \hlkwb{<-} \hlkwd{t.test}\hlstd{(x1, x2)}\hlopt{$}\hlstd{p.value}
    \hlstd{p2} \hlkwb{<-} \hlkwd{wilcox.test}\hlstd{(x1, x2)}\hlopt{$}\hlstd{p.value}
    \hlkwd{c}\hlstd{(}\hlkwc{t} \hlstd{= p1,} \hlkwc{wilcox}\hlstd{=p2)}
\hlstd{\}}

\hlkwd{do1rep}\hlstd{(}\hlnum{30}\hlstd{, rnorm,} \hlnum{0}\hlstd{)}
\end{alltt}
\begin{verbatim}
##         t    wilcox 
## 0.4054286 0.5133481
\end{verbatim}
\end{kframe}
\end{knitrout}

Now we can check the empirical rejection rates of the tests in a
simulation study.
\begin{knitrout}
\definecolor{shadecolor}{rgb}{0.969, 0.969, 0.969}\color{fgcolor}\begin{kframe}
\begin{alltt}
\hlstd{nrep} \hlkwb{<-} \hlnum{1000}
\hlstd{sim} \hlkwb{<-} \hlkwd{replicate}\hlstd{(nrep,} \hlkwd{do1rep}\hlstd{(n, rnorm,} \hlnum{0}\hlstd{))}
\hlkwd{rowMeans}\hlstd{(sim} \hlopt{<} \hlnum{.05}\hlstd{)}
\end{alltt}
\begin{verbatim}
##      t wilcox 
##  0.055  0.058
\end{verbatim}
\begin{alltt}
\hlcom{## put them into a function for ease of accessing}
\hlstd{empRejRate} \hlkwb{<-} \hlkwa{function}\hlstd{(}\hlkwc{nrep}\hlstd{,} \hlkwc{n}\hlstd{,} \hlkwc{datagen}\hlstd{,} \hlkwc{delta} \hlstd{=} \hlnum{0}\hlstd{,} \hlkwc{alpha} \hlstd{=} \hlnum{.05}\hlstd{) \{}
    \hlstd{sim} \hlkwb{<-} \hlkwd{replicate}\hlstd{(nrep,} \hlkwd{do1rep}\hlstd{(n, datagen, delta))}
    \hlkwd{rowMeans}\hlstd{(sim} \hlopt{<} \hlstd{alpha)}
\hlstd{\}}

\hlcom{## normal population}
\hlkwd{empRejRate}\hlstd{(nrep, n, rnorm,} \hlnum{0}\hlstd{)}
\end{alltt}
\begin{verbatim}
##      t wilcox 
##  0.056  0.051
\end{verbatim}
\begin{alltt}
\hlkwd{empRejRate}\hlstd{(nrep, n, rnorm,} \hlnum{0.5}\hlstd{)}
\end{alltt}
\begin{verbatim}
##      t wilcox 
##  0.501  0.489
\end{verbatim}
\begin{alltt}
\hlcom{## Cauchy population}
\hlkwd{empRejRate}\hlstd{(nrep, n, rcauchy,} \hlnum{0}\hlstd{)}
\end{alltt}
\begin{verbatim}
##      t wilcox 
##  0.020  0.047
\end{verbatim}
\begin{alltt}
\hlkwd{empRejRate}\hlstd{(nrep, n, rcauchy,} \hlnum{0.5}\hlstd{)}
\end{alltt}
\begin{verbatim}
##      t wilcox 
##  0.032  0.171
\end{verbatim}
\end{kframe}
\end{knitrout}

We can draw power curves to compare two tests.
\begin{knitrout}
\definecolor{shadecolor}{rgb}{0.969, 0.969, 0.969}\color{fgcolor}\begin{kframe}
\begin{alltt}
\hlstd{delta} \hlkwb{<-} \hlkwd{seq}\hlstd{(}\hlnum{0}\hlstd{,} \hlnum{1}\hlstd{,} \hlkwc{by} \hlstd{=} \hlnum{.2}\hlstd{)}
\hlcom{## normal distribuion}
\hlstd{rejrate} \hlkwb{<-} \hlkwd{sapply}\hlstd{(delta,} \hlkwa{function}\hlstd{(}\hlkwc{x}\hlstd{)} \hlkwd{empRejRate}\hlstd{(nrep, n, rnorm, x))}
\hlkwd{plot}\hlstd{(delta, rejrate[}\hlstr{"t"}\hlstd{, ],} \hlkwc{type} \hlstd{=} \hlstr{"l"}\hlstd{,} \hlkwc{ylab} \hlstd{=} \hlstr{"emporical rejection rate"}\hlstd{,} \hlkwc{ylim} \hlstd{=} \hlkwd{c}\hlstd{(}\hlnum{0}\hlstd{,} \hlnum{1}\hlstd{))}
\hlkwd{lines}\hlstd{(delta, rejrate[}\hlstr{"wilcox"}\hlstd{, ],} \hlkwc{lty} \hlstd{=} \hlnum{2}\hlstd{,} \hlkwc{col} \hlstd{=} \hlstr{"blue"}\hlstd{)}
\hlkwd{abline}\hlstd{(}\hlnum{.05}\hlstd{,} \hlnum{0}\hlstd{)}
\end{alltt}
\end{kframe}
\includegraphics[width=\maxwidth]{figure/hypo-twosample-power-1} 
\begin{kframe}\begin{alltt}
\hlcom{## Cauchy distribuion}
\hlstd{rejrate} \hlkwb{<-} \hlkwd{sapply}\hlstd{(delta,} \hlkwa{function}\hlstd{(}\hlkwc{x}\hlstd{)} \hlkwd{empRejRate}\hlstd{(nrep, n, rcauchy, x))}
\hlkwd{plot}\hlstd{(delta, rejrate[}\hlstr{"t"}\hlstd{, ],} \hlkwc{type} \hlstd{=} \hlstr{"l"}\hlstd{,} \hlkwc{ylab} \hlstd{=} \hlstr{"emporical rejection rate"}\hlstd{,} \hlkwc{ylim} \hlstd{=} \hlkwd{c}\hlstd{(}\hlnum{0}\hlstd{,} \hlnum{1}\hlstd{))}
\hlkwd{lines}\hlstd{(delta, rejrate[}\hlstr{"wilcox"}\hlstd{, ],} \hlkwc{lty} \hlstd{=} \hlnum{2}\hlstd{,} \hlkwc{col} \hlstd{=} \hlstr{"blue"}\hlstd{)}
\hlkwd{abline}\hlstd{(}\hlnum{.05}\hlstd{,} \hlnum{0}\hlstd{)}
\end{alltt}
\end{kframe}
\includegraphics[width=\maxwidth]{figure/hypo-twosample-power-2} 
\end{knitrout}

Sometimes, we don't need to resort to tests based on large sample
theory. Consider the setting where we compare the two arms of
treatment in a clinical trial: treatment versus placebo. Suppose that
we want to show that the treatment leads to an increase in an outcome
of interest. Let $\mu_0$ and $\mu_1$ denote the mean of the outcome
under placebo and treatment, respectively. The hypotheses to be tested
are
\[
  H_0: \mu_1 = \mu_0,
  \quad \mbox{vs} \quad
  H_1: \mu_1 > \mu_0.
\]


Under $H_0$, the labels of the treatment arms are exchangeable.
Consider test statistics $T$, the difference in the two sample means.
\begin{knitrout}
\definecolor{shadecolor}{rgb}{0.969, 0.969, 0.969}\color{fgcolor}\begin{kframe}
\begin{alltt}
\hlstd{xpooled} \hlkwb{<-} \hlkwd{c}\hlstd{(x1, x2)}
\hlstd{xperm} \hlkwb{<-} \hlkwd{sample}\hlstd{(xpooled,} \hlkwc{size} \hlstd{=} \hlkwd{length}\hlstd{(xpooled),} \hlkwc{replace} \hlstd{=} \hlnum{FALSE}\hlstd{)}
\hlstd{xd} \hlkwb{<-}  \hlkwd{mean}\hlstd{(xperm[(n1} \hlopt{+} \hlnum{1}\hlstd{)}\hlopt{:}\hlstd{(n1} \hlopt{+} \hlstd{n2)])} \hlopt{-} \hlkwd{mean}\hlstd{(xperm[}\hlnum{1}\hlopt{:}\hlstd{n1])}
\hlcom{## put in a function}
\hlstd{myPermTest} \hlkwb{<-} \hlkwa{function}\hlstd{(}\hlkwc{x1}\hlstd{,} \hlkwc{x2}\hlstd{,} \hlkwc{nperm} \hlstd{=} \hlnum{1000}\hlstd{) \{}
    \hlstd{n1} \hlkwb{<-} \hlkwd{length}\hlstd{(x1)}
    \hlstd{n2} \hlkwb{<-} \hlkwd{length}\hlstd{(x2)}
    \hlstd{stat} \hlkwb{<-} \hlkwd{mean}\hlstd{(x2)} \hlopt{-} \hlkwd{mean}\hlstd{(x1)}
    \hlstd{xpooled} \hlkwb{<-} \hlkwd{c}\hlstd{(x1, x2)}
    \hlstd{stat.sim} \hlkwb{<-} \hlkwd{replicate}\hlstd{(nperm, \{}
        \hlstd{xperm} \hlkwb{<-} \hlkwd{sample}\hlstd{(xpooled,} \hlkwc{size} \hlstd{=} \hlkwd{length}\hlstd{(xpooled),} \hlkwc{replace} \hlstd{=} \hlnum{FALSE}\hlstd{)}
        \hlstd{xd} \hlkwb{<-} \hlkwd{mean}\hlstd{(xperm[(n1} \hlopt{+} \hlnum{1}\hlstd{)}\hlopt{:}\hlstd{(n1} \hlopt{+} \hlstd{n2)])} \hlopt{-} \hlkwd{mean}\hlstd{(xperm[}\hlnum{1}\hlopt{:}\hlstd{n1])}
    \hlstd{\})}
    \hlstd{p.value}  \hlkwb{<-} \hlkwd{mean}\hlstd{(stat.sim} \hlopt{>=} \hlstd{stat)}
    \hlstd{p.value}
\hlstd{\}}

\hlstd{delta} \hlkwb{<-} \hlnum{1}
\hlstd{x1} \hlkwb{<-} \hlkwd{rgamma}\hlstd{(n,} \hlkwc{shape} \hlstd{=} \hlnum{2}\hlstd{,} \hlkwc{scale} \hlstd{=} \hlnum{2}\hlstd{)}
\hlstd{x2} \hlkwb{<-} \hlkwd{rgamma}\hlstd{(n,} \hlkwc{shape} \hlstd{=} \hlnum{2}\hlstd{,} \hlkwc{scale} \hlstd{=} \hlnum{2}\hlstd{)} \hlopt{+} \hlstd{delta}
\hlkwd{myPermTest}\hlstd{(x1, x2)}
\end{alltt}
\begin{verbatim}
## [1] 0.12
\end{verbatim}
\begin{alltt}
\hlcom{## try a simulation study}
\hlstd{do1rep} \hlkwb{<-} \hlkwa{function}\hlstd{(}\hlkwc{n}\hlstd{,} \hlkwc{datagen}\hlstd{,} \hlkwc{delta} \hlstd{=} \hlnum{0}\hlstd{) \{}
    \hlstd{x1} \hlkwb{<-} \hlkwd{datagen}\hlstd{(n)}
    \hlstd{x2} \hlkwb{<-} \hlkwd{datagen}\hlstd{(n)} \hlopt{+} \hlstd{delta}
    \hlkwd{myPermTest}\hlstd{(x1, x2)}
\hlstd{\}}

\hlstd{sim} \hlkwb{<-} \hlkwd{replicate}\hlstd{(}\hlnum{200}\hlstd{,} \hlkwd{do1rep}\hlstd{(n, rnorm,} \hlkwc{delta} \hlstd{=} \hlnum{.5}\hlstd{))}
\hlkwd{mean}\hlstd{(sim} \hlopt{<} \hlnum{.05}\hlstd{)}
\end{alltt}
\begin{verbatim}
## [1] 0.645
\end{verbatim}
\begin{alltt}
\hlstd{sim} \hlkwb{<-} \hlkwd{replicate}\hlstd{(}\hlnum{200}\hlstd{,} \hlkwd{do1rep}\hlstd{(n, rcauchy,} \hlkwc{delta} \hlstd{=} \hlnum{1}\hlstd{))}
\hlkwd{mean}\hlstd{(sim} \hlopt{<} \hlnum{.05}\hlstd{)}
\end{alltt}
\begin{verbatim}
## [1] 0.16
\end{verbatim}
\end{kframe}
\end{knitrout}


We can test on data generated from a distribution with flexible
shapes, the generalized Tukey's lambda family, which is defined in
terms of the inverse of the distribution function.
\begin{knitrout}
\definecolor{shadecolor}{rgb}{0.969, 0.969, 0.969}\color{fgcolor}\begin{kframe}
\begin{alltt}
\hlstd{rgtl} \hlkwb{<-} \hlkwa{function}\hlstd{(}\hlkwc{n}\hlstd{,} \hlkwc{lambda1} \hlstd{=} \hlnum{0}\hlstd{,} \hlkwc{lambda2} \hlstd{=} \hlnum{1}\hlstd{,} \hlkwc{lambda3} \hlstd{=} \hlnum{0}\hlstd{,} \hlkwc{lambda4} \hlstd{=} \hlnum{1}\hlstd{) \{}
    \hlstd{u} \hlkwb{<-} \hlkwd{runif}\hlstd{(n)}
    \hlstd{lambda1} \hlopt{+} \hlstd{(u} \hlopt{^} \hlstd{lambda3} \hlopt{-} \hlstd{(}\hlnum{1} \hlopt{-} \hlstd{u)} \hlopt{^} \hlstd{lambda4)} \hlopt{/} \hlstd{lambda2}
\hlstd{\}}

\hlstd{sim} \hlkwb{<-} \hlkwd{replicate}\hlstd{(}\hlnum{200}\hlstd{,}
                 \hlkwd{do1rep}\hlstd{(n,}
                        \hlkwa{function}\hlstd{(}\hlkwc{n}\hlstd{)} \hlkwd{rgtl}\hlstd{(n,} \hlkwc{lambda3} \hlstd{=} \hlnum{1.5}\hlstd{,} \hlkwc{lambda4} \hlstd{=} \hlnum{5.8}\hlstd{),}
                        \hlkwc{delta} \hlstd{=} \hlnum{.2}\hlstd{))}
\end{alltt}
\end{kframe}
\end{knitrout}
